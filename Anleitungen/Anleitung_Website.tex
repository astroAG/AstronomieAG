
\documentclass[12pt,a4paper,notitlepage,onecolumn,portrait,oneside, , , ]{scrartcl}
\usepackage{ngerman, enumitem, amsmath, amssymb, float, hyperref}
\usepackage[utf8]{inputenc}
\usepackage[T1]{fontenc}

\begin{document}

\title{Anleitung: GitHub}
	\subtitle{Astro-AG der Liebigschule}
\author{Marcel Wunderlich, überarbeitet von Patrick Hemberger}

\maketitle

\section{Code}
\subsection{HTML}
Im Prinzip ganz normaler Text, der durch Tags formatiert wird. Tags sind zum Beispiel \textbf{<html>, <head>, <body>}, etc. In der Regel haben alle Tags ein öffnendes und ein schließendes Tag (gleicher Name mit Slash am Anfang), sodass ein Bereich eingeschlossen wird: \textbf{<html></html>, <h1></h1>}, etc. Außerdem gibt es noch Tags mit zusätzlichen Parametern (die nur beim öffnenden Tag stehen): \textbf{<a href=“meineSeite.html“></a>, <font color=“green“></font>}, etc.\\
\textbf{<!-- -->} sind übrigens Kommentare. Wenn ihr also \textbf{<!--Hier kommen die Bilder hin-->} schreibt, ist der Text nur im Code (via Editor) zu sehen, aber im Browser unsichtbar. Das kann helfen, sich Notizen im Code zu machen um sich zum Beispiel besser zurecht zu finden.\\
Zum Lernen und insbesondere zum Nachschlagen bietet sich die SelfHTML-Referenz an:\\
\url{http://de.selfhtml.org/} \\
Eine gute Übersicht der Farben mit zugehörigen Hex-Codes findet man unter: \\
\url{http://tomheller.de/theholycymbal/html-farben.html} \\

\subsection{PHP}


%Seitenumbruch, wenn keine volle Aufgabe mehr aufs Blatt passt.
%\pagebreak

%Tabellenvorlage
%\begin{center}
%\begin{tabular}{|ll|...}
%vor erster Zeile obere Umrandung
%\hline
%Überschriften der Spalten:
%US1 & US2 ...\\ \hline
%Spalteninhalte:
%IS1 & IS2  \\ 
%nach letzter Zeile untere Umrandung
%\hline
%\end{tabular}
%\end{center}

%horizontal ausgerichtete mathematische Umgebung 
%\begin{align*}
%	a &= b \\
%	%
%\end{align*}

\end{document}
